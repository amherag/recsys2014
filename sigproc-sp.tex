% THIS IS SIGPROC-SP.TEX - VERSION 3.1
% WORKS WITH V3.2SP OF ACM_PROC_ARTICLE-SP.CLS
% APRIL 2009
%
% It is an example file showing how to use the 'acm_proc_article-sp.cls' V3.2SP
% LaTeX2e document class file for Conference Proceedings submissions.
% ----------------------------------------------------------------------------------------------------------------
% This .tex file (and associated .cls V3.2SP) *DOES NOT* produce:
%       1) The Permission Statement
%       2) The Conference (location) Info information
%       3) The Copyright Line with ACM data
%       4) Page numbering
% ---------------------------------------------------------------------------------------------------------------
% It is an example which *does* use the .bib file (from which the .bbl file
% is produced).
% REMEMBER HOWEVER: After having produced the .bbl file,
% and prior to final submission,
% you need to 'insert'  your .bbl file into your source .tex file so as to provide
% ONE 'self-contained' source file.
%
% Questions regarding SIGS should be sent to
% Adrienne Griscti ---> griscti@acm.org
%
% Questions/suggestions regarding the guidelines, .tex and .cls files, etc. to
% Gerald Murray ---> murray@hq.acm.org
%
% For tracking purposes - this is V3.1SP - APRIL 2009

\documentclass{acm_proc_article-sp}

\begin{document}


\title{Sentiment Analysis using Keyboard and Mouse Dynamics for the
  Sequencing of Computer Programming Exercises}

%
% You need the command \numberofauthors to handle the 'placement
% and alignment' of the authors beneath the title.
%
% For aesthetic reasons, we recommend 'three authors at a time'
% i.e. three 'name/affiliation blocks' be placed beneath the title.
%
% NOTE: You are NOT restricted in how many 'rows' of
% "name/affiliations" may appear. We just ask that you restrict
% the number of 'columns' to three.
%
% Because of the available 'opening page real-estate'
% we ask you to refrain from putting more than six authors
% (two rows with three columns) beneath the article title.
% More than six makes the first-page appear very cluttered indeed.
%
% Use the \alignauthor commands to handle the names
% and affiliations for an 'aesthetic maximum' of six authors.
% Add names, affiliations, addresses for
% the seventh etc. author(s) as the argument for the
% \additionalauthors command.
% These 'additional authors' will be output/set for you
% without further effort on your part as the last section in
% the body of your article BEFORE References or any Appendices.

\numberofauthors{3} %  in this sample file, there are a *total*
% of EIGHT authors. SIX appear on the 'first-page' (for formatting
% reasons) and the remaining two appear in the \additionalauthors section.
%
\author{
% You can go ahead and credit any number of authors here,
% e.g. one 'row of three' or two rows (consisting of one row of three
% and a second row of one, two or three).
%
% The command \alignauthor (no curly braces needed) should
% precede each author name, affiliation/snail-mail address and
% e-mail address. Additionally, tag each line of
% affiliation/address with \affaddr, and tag the
% e-mail address with \email.
%
% 1st. author
\alignauthor
Amaury Hernandez-Aguila\\
       \affaddr{Tijuana Institute of Technology}\\
       \affaddr{Calzada Tecnologico s/n, Tomas Aquino}\\
       \affaddr{Tijuana, Mexico}\\
       \email{amherag@tectijuana.edu.mx}
% 2nd. author
\alignauthor
Mario Garcia-Valdez\\
       \affaddr{Tijuana Institute of Technology}\\
       \affaddr{Calzada Tecnologico s/n, Tomas Aquino}\\
       \affaddr{Tijuana, Mexico}\\
       \email{mario@tectijuana.edu.mx}
% 3rd. author
\alignauthor
Alejandra Mancilla\\
       \affaddr{Tijuana Institute of Technology}\\
       \affaddr{Calzada Tecnologico s/n, Tomas Aquino}\\
       \affaddr{Tijuana, Mexico}\\
       \email{alejandra.mancilla@gmail.com}
\and  % use '\and' if you need 'another row' of author names
% 4th. author
\alignauthor
% 5th. author
\alignauthor
% 6th. author
\alignauthor
}
% There's nothing stopping you putting the seventh, eighth, etc.
% author on the opening page (as the 'third row') but we ask,
% for aesthetic reasons that you place these 'additional authors'
% in the \additional authors block, viz.

% Just remember to make sure that the TOTAL number of authors
% is the number that will appear on the first page PLUS the
% number that will appear in the \additionalauthors section.

\maketitle
\begin{abstract}
This work presents a method based on keystroke and mouse dynamics for
the analysis of a student's sentiments while interacting with an
intelligent tutoring system called Protoboard, which focuses on the
teaching of computer programming. The data gathered by the keystroke
and mouse dynamics is used to recommend a sequence of programming
exercises for a student that is interacting with the system. This
sequence of exercises affects the student's mental states during
the course of the programming lessons and exercises, with the purpose
of enhancing the learning experience of the student. The method
focuses on maximising or minimising six mental states: frustration,
boredom, relaxation, distraction, concentration, and excitement. For
the prediction of these mental states, neural networks classify a
student according to their keyboard and mouse dynamics into different
degrees of the mental states. These degrees are used for a recommender
system to determine a better sequencing of the exercises to be
presented to the student. A prototype of the system has been
developed, and is currently being tested.
\end{abstract}
%169 words

% A category with the (minimum) three required fields
\category{H.4}{Information Systems Applications}{Miscellaneous}
%A category including the fourth, optional field follows...
\category{D.2.8}{Software Engineering}{Metrics}[complexity measures, performance measures]

\terms{Theory}

\keywords{ACM proceedings, \LaTeX, text tagging} % NOT required for Proceedings

\section{Introduction}

\subsection{Intelligent Tutoring Systems}

The prototype that is being developed falls under the category of
Intelligent Tutoring Systems (ITS). An ITS is any software which
objective is to facilitate a student's learning through the use of
different tools, such as natural language processing, semantic web or
machine learning. In this case, Protoboard uses machine learning
techniques to predict a student's mental states and emotions, and use
these emotions as input to a recommender system. This is achieved
using neural networks as the classification model, and keystroke and
mouse dynamics as the input to this model.

The recommendation is performed over an implementation of the Simple
Sequencing (SS) specification. SS allows us to establish a precondition
rule on each of the learning objects in a learning
activity. Precondition rules are defined to restrain a learning object
from being shown to the user, and allow other objects to be shown.

%144 words

\subsection{Affective Computing}
One of the areas of study that Protoboard is leveraging is Affective
Computing, a branch of computer science which studies how to interpret
and emulate human emotions. The platform recognises how a student was
feeling during their interaction with a learning object, such as a
video, a programming exercise or a quiz, and these predictions can
then be used to recommend what the next learning object will be, in
order to maximise or minimise an emotion.

The resulting predicted values of a user's mental states and emotions
can be used for other content adaptation tasks in the system, e.g.,
estimating if a student needs more exercises of certain learning style
(visual, verbal, aural, etc.), or as input to other processes, as in
the case of a module for predicting if a student needs more hints, or
a module that determines if a student is cheating in a test.

%149 words

\subsection{Simple Sequencing}





\balancecolumns
% That's all folks!
\end{document}
